%% ================================================================================
%% This LaTeX file was created by AbiWord.                                         
%% AbiWord is a free, Open Source word processor.                                  
%% More information about AbiWord is available at http://www.abisource.com/        
%% ================================================================================

\documentclass[a4paper,portrait,12pt]{article}
\usepackage[latin1]{inputenc}
\usepackage{calc}
\usepackage{setspace}
\usepackage{fixltx2e}
\usepackage{graphicx}
\usepackage{multicol}
\usepackage[normalem]{ulem}
%% Please revise the following command, if your babel
%% package does not support fr-CA
\usepackage[frenchb]{babel}
\usepackage{color}
\usepackage{hyperref}
 
\begin{document}


\begin{flushleft}
Lab1.
\end{flushleft}


\begin{flushleft}
Exercices:
\end{flushleft}


\begin{flushleft}
1. D\'{e}terminer la puissance de chaque chiffre pour un nombre XXXXX6 de 5 chiffres en base 6.
\end{flushleft}


\begin{flushleft}
2. Utiliser ce r\'{e}sultat pour convertir le nombre 245316 en d\'{e}cimal.
\end{flushleft}


\begin{flushleft}
3. Convertir les nombres suivants d'hexad\'{e}cimal \`{a} d\'{e}cimal:
\end{flushleft}


\begin{flushleft}
a) 4E16
\end{flushleft}


\begin{flushleft}
b) 3D716
\end{flushleft}


\begin{flushleft}
c) 3D7016
\end{flushleft}


\begin{flushleft}
4. Combien de bits faut-il pour repr\'{e}senter le nombre d\'{e}cimal 3175000?
\end{flushleft}


\begin{flushleft}
5. Combien d'octets faudra-t-il pour stocker ce nombre?
\end{flushleft}


\begin{flushleft}
6. Faire \`{a} la main les calculs suivants (sans convertir \`{a} une autre base, tel que d\'{e}cimal):
\end{flushleft}


\begin{flushleft}
a)
\end{flushleft}


\begin{flushleft}
2AB316
\end{flushleft}


\begin{flushleft}
+ 35DC16
\end{flushleft}


\begin{flushleft}
b)
\end{flushleft}


\begin{flushleft}
1FF916
\end{flushleft}


\begin{flushleft}
+ F716
\end{flushleft}


\begin{flushleft}
c)
\end{flushleft}


110100112


+ 100010102


\begin{flushleft}
d)
\end{flushleft}


11012


× 101 2


\begin{flushleft}
e)
\end{flushleft}


110112


× 1011 2


\begin{flushleft}
7. Convertir les nombres binaires suivants \`{a} l'hexad\'{e}cimal:
\end{flushleft}


\begin{flushleft}
a) 1011011101110102
\end{flushleft}


\begin{flushleft}
b) 11111111111100012
\end{flushleft}


\begin{flushleft}
c) 11111111011112
\end{flushleft}


\begin{flushleft}
d) 11000110001100012
\end{flushleft}


\begin{flushleft}
8. Convertir les nombres hexad\'{e}cimaux suivants au binaire:
\end{flushleft}


\begin{flushleft}
a) 4F6A16
\end{flushleft}


\begin{flushleft}
b) 990216
\end{flushleft}


\begin{flushleft}
c) A3AB16
\end{flushleft}


\begin{flushleft}
d) 100016
\end{flushleft}





\newpage



\end{document}
